Sull'applicazione sono stati eseguiti diversi \textit{test} con risultati abbastanza buoni. Il fatto che ci voglia una finestra minima di 3 secondi per avere almeno un'immagine predetta, fa si che bisogni registrare un bel po di canzone per avere un risultato più accurato.\\
Alcune predizioni sbagliate potrebbero dipendere dal fatto che ci possa essere confusione tra le classi. Spesso le canzoni appartengono a più generi e sottogeneri che si somigliano tra loro e quindi questo potrebbe portare ad una predizione errata.\\
Espandere il campione originale è stato determinante per avere risultati migliori, nonostante ciò anche 1000 spettrogrammi per genere potrebbero essere un campione molto piccolo poiché stiamo addestrando dei modelli da zero. Un set di dati ancora più grande dovrebbe migliorarne i risultati.\\
Sarebbe interessante indagare meglio sul perchè il set di dati \textit{FMA} non abbia raggiunto i risultati del set di dati \textit{GTZAN}. Forse il primo è più impegnativo.\\
Una cosa certa è che l'utilizzo dell'albero decisionale ha contribuito a migliorare l'accuratezza del modello.\\
\newline
Nonostante tutto, il progetto realizzato è stato molto soddisfacente. Grazie a questa materia, prima con \textit{Sistemi Digitali M}, e poi con la corrispondente \textit{Attività Progettuale} si è potuto imparare ed esplorare un mondo bellissimo e vasto come quello del \textit{Machine Learning}.\\
\newline
A fini dimostrativi, viene fornito insieme alla documentazione un breve video dell'applicazione in funzione.