Come mai prima d'ora, il web è diventato un luogo di condivisione di lavori creativi, come la musica, tra una comunità globale di artisti e amanti dell'arte. Sebbene le raccolte di musica precedono la nascita del Web, esso ha consentito raccolte su scala molto più ampia. Mentre prima le persone possedevano i dischi in vinile o i CD, al giorno d'oggi hanno accesso istantaneo a tutti i contenuti musicali pubblicati tramite piattaforme di streaming online come Spotify, iTunes, Youtube ecc.\\ Un aumento così drastico delle dimensioni delle raccolte musicali ha dato vita a due sfide:
\begin{itemize}
\item la necessità di organizzare automaticamente una raccolta (poiché utenti ed editori non possono più gestirle manualmente);
\item la necessità di consigliare automaticamente nuove canzoni a un utente che conosce le proprie abitudini di ascolto;
\end{itemize}
Un compito fondamentale in entrambe queste sfide è essere in grado di raggruppare le canzoni in categorie semantiche. I generi musicali sono categorie che sono sorte attraverso una complessa interazione di culture, artisti e forze di mercato per caratterizzare le somiglianze tra le composizioni e organizzare le raccolte musicali. Eppure i confini tra i generi rimangono ancora confusi, rendendo il problema del riconoscimento del genere musicale un compito non banale.\\ Il progetto ha l'obiettivo di riconoscere in modo automatico il genere di un brano musicale di cui è disponibile solo una registrazione tramite l'utilizzo di reti neurali convoluzionali (\textit{CNN}). Il tutto funziona tramite l'ausilio di un semplice \textit{smartphone}. \\

La relazione è articolata come segue:
\begin{itemize}
	\item nel \textbf{capitolo 1} viene presentato il dataset utilizzato;
	\item nel \textbf{capitolo 2} viene descritto il modello della rete, l'addestramento che è stato eseguito e la sua accuratezza;
	\item nel \textbf{capitolo 3} viene realizzata l'implementazione sul dispositivo \textit{embedded}, gli \textit{smartphone};
	\item il \textbf{capitolo 4} conclude la relazione presentando i risultati ottenuti, gli obiettivi raggiunti ed eventuali problematiche che potrebbero essere risolte in futuro.
\end{itemize}
La relazione è stata scritta come un \textit{diario} mettendo in risalto tutti i passaggi, i tentativi e i problemi che si sono verificati.\\
\newline
Il codice, i modelli, i \textit{log} e l'applicazione di tutto il progetto possono essere reperiti nella seguente repository git: \href{https://github.com/it9tst/music-genre-recognition}{https://github.com/it9tst/music-genre-recognition}