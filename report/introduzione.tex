La musica ha assunto un ruolo di primo piano nella storia dell'uomo. Forse è la voglia di esprimere i propri sentimenti e le proprie emozioni che hanno obbligato l'essere umano a comporre sempre nuovi brani. Senza alcun dubbio, la chitarra è uno degli strumenti più usati. Per esempio, quando si va in campeggio e alla sera ci si siede davanti al falò, sono proprio le sue note a riempire l'aria.\\ Il tema libero lasciato dai professori ha permesso di approfondire ma soprattutto conoscere meglio il vastissimo campo della musica.  \\ Il progetto ha l'obiettivo di trascrivere in modo automatico le tablature per chitarra tramite l'utilizzo di reti neurali convoluzionali (\textit{CNN}) in modo che anche chi non è in grado di suonare questo strumento lo possa fare. Il tutto funziona tramite l'ausilio di un semplice \textit{smartphone}. \\

La relazione è articolata come segue:
\begin{itemize}
	\item nel \textbf{capitolo 1} viene descritta brevemente la chitarra;
	\item nel \textbf{capitolo 2} viene introdotto il dominio in cui lavoreremo;
	\item nel \textbf{capitolo 3} viene descritto il modello della nostra rete, l'addestramento che è stato eseguito e la sua accuratezza;
	\item nel \textbf{capitolo 4} viene realizzata l'implementazione sul dispositivo \textit{embedded}, nel nostro caso sono gli \textit{smartphone};
	\item il \textbf{capitolo 5} conclude la relazione presentando i risultati ottenuti, gli obiettivi raggiunti ed eventuali problematiche che potrebbero essere risolte in futuro.
\end{itemize}
La relazione è stata scritta come un \textit{diario} mettendo in risalto tutti i passaggi, i tentativi e i problemi che si sono verificati.\\
\newline
Il codice, il modello e i \textit{log} di tutto il progetto possono essere reperiti nella seguente repository git: \href{https://github.com/it9tst/tab-writer}{https://github.com/it9tst/tab-writer} 
 